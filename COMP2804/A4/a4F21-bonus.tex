\documentclass[12pt]{article}

\usepackage{hyperref}
\usepackage{fullpage}
\usepackage{amssymb}
\usepackage{fancybox}
\usepackage{amsmath}
\usepackage{paralist}
\usepackage{xcolor}

\usepackage{verbatim}

\newcommand{\IE}{\mathbb{E}}
\newcommand{\Mystery}{\textsf{\sc Mystery}}
\newcommand{\EULER}{\textsf{\sc Euler}}
\newcommand{\TOTAL}{\mathord{\it total}}
\newcommand{\DEG}{\mathord{\it deg}}


%%%%%%%%%%%%%% Capsule %%%%%%%%%%%%%%%%%%%%%%%%%%%%%%%%%%%%%%%%%%%
\newcommand{\capsule}[2]{\vspace{0.5em}
  \shadowbox{%
    \begin{minipage}{.90\linewidth}%
      \textbf{#1:}~#2%
    \end{minipage}}
  \vspace{0.5em} }
%%%%%%%%%%%%%%%%%%%%%%%%%%%%%%%%%%%%%%%%%%%%%%%%%%%%%%%%%%%%%%%%%%

\newcounter{ques}
\newenvironment{question}{\stepcounter{ques}{\noindent\bf Question \arabic{ques}:}}{\vspace{5mm}}
\newcommand{\solution}[1]{{\color{blue}\textbf{Solution: }#1}}


\begin{document}

\begin{center} \Large\bf
COMP 2804 --- Assignment 4
\end{center}

\noindent {\bf Due:} Sunday December 5, 11:59 pm.

\vspace{0.5em}

\noindent {\bf Assignment Policy:}
\begin{itemize}
\item Your assignment must be submitted as one single PDF file through
      cuLearn.
\item {\bf Late assignments will not be accepted.}
\item You are encouraged to collaborate on assignments, but at the level
      of discussion only. When writing your solutions, you must do so
      in your own words.
\item Past experience has shown conclusively that those who do not put
      adequate effort into the assignments do not learn the material and
      have a probability near 1 of doing poorly on the exams.
\item When writing your solutions, you must follow the guidelines below.
      \begin{itemize}
      \item You must justify your answers.
      \item The answers should be concise, clear and neat.
      \item When presenting proofs, every step should be justified.
      \item Each part of each question is worth 2 marks, except for 4.1 (4 marks) and 4.2 (6 marks)
      \end{itemize}
\end{itemize}

\begin{question}
\begin{itemize}
\item Write your name and student number.
\end{itemize}
\end{question}



\begin{question}
	\href{https://en.wikipedia.org/wiki/Craps}{Craps} is a game where you roll two dice and bet on the outcome. There are many ways to play, but we will focus on one specific scenario. On your "come-out" roll you roll a $9$ on two dice. Now the game is this: you must keep rolling 2 dice until you either roll $9$ again, in which case you win, or you roll a $7$, in which case the house wins (you lose).
	\begin{enumerate}
		\item \begin{enumerate}
			\item \label{a} Find the probability that you win.
			\item \label{b} Find the probability that the house wins.
		\end{enumerate} 
		 Show all your work for both parts. Observe that the probabilities from \ref{a} and \ref{b} added together should total $1$, but you may NOT use this fact to determine your answers.
		
		
		
		\item Assume you have bet $\$10$. You roll a $9$ on your "come-out" roll, same as above. If you win the casino pays you $\$10$. If you lose the casino takes your $\$10$ bet. What is your expected winnings or losings (i.e., the expected value) of this round of craps?
		
	
	\end{enumerate}
	  
\end{question}
\newpage
\begin{question}
	You are playing a board game about pirates that uses special six-sided fair dice. Three sides of each die has a zero on it, while the other three sides have the numbers $1$ through $3$ (this represents damage done by your ship's cannons). Your ship has five cannons, so to attack you roll $5$ of these six-sided dice.  Let 
	\begin{eqnarray*} 
		A & = & \mbox{The sum of the values of the $5$ dice.} \\ 
		B & = & \mbox{The number of times a zero is showing.} \\ 
	\end{eqnarray*} 
	
	\begin{enumerate}
		\item What is $E(A)$ and E$(B)$?
		
	
		\item Are $A$ and $B$ independent random variables?
		
	
	\end{enumerate}
\end{question}




\begin{question}
	When FX and his girlfriend XF have a child, this child is a boy with 
	probability $1/2$ and a girl with probability $1/2$, independently of 
	the sex of their other children. FX and XF stop having children as soon 
	as they have three girls or two boys. 
	
	
	Consider the random variables 
	\begin{eqnarray*} 
		C & = & \mbox{the number of children that FX and XF have,} \\ 
		G & = & \mbox{the number of girls that FX and XF have.} \\ 
		B & = & \mbox{the number of boys that FX and XF have.} \\ 
	\end{eqnarray*} 

\begin{enumerate}
	\item Determine the expected values $E(C)$ and $E(G)$
	
	\item We are told that the last child FX and XF had was a boy. Determine the expected values $E(B)$ and $E(G)$ and $E(C)$.

\end{enumerate}
	
	
	 
\end{question}



\begin{question}
	In the \href{https://dnd.wizards.com/}{Dungeons and Dragons} question from last assignment we talked about how to roll for stats (that is, we take the sum of $3$ six-sided dice). It was briefly mentioned that to get higher stats, we can roll $4$ six-sided dice and take the sum of the $3$ highest dice. In this question we will compare the expected outcome from both of these techniques.
	\begin{enumerate}
		\item\label{threedice} Determine the expected value of the sum of rolling $3$ fair six-sided dice.
		\item What is the size of the sample space $S$ of all possible dice rolls with $4$ dice?
	\end{enumerate}

		Let $d_1,d_2,d_3,d_4$ be random variables corresponding to the values of $4$ six-sided dice after being rolled. Let $Y$ be the value of the three highest dice. That is, $Y =\left(\sum_{i=1}^4 d_i \right)- \min\{d_1,d_2,d_3,d_4\}$. Thus our goal is to find $E(Y)$ and compare it to the answer from \ref{threedice}.
		
		Let the random variable $X_i$ be the sum of the highest $3$ dice if the lowest die is equal to $i$, that is $\min\{d_1,d_2,d_3,d_4\}=i$.  Let $X_i =0$ if the lowest die is not equal to $i$.  Let $A_i$ be the event that the lowest die out of $d_1,d_2,d_3,$ and $d_4$ is equal to $i$. 
	\begin{enumerate}\setcounter{enumi}{2}	
		\item In a few sentences explain why $$\sum_{\omega \in S} Y(\omega)= \sum_{i=1}^6 \sum_{\omega \in S} X_i(\omega) = \sum_{i=1}^6 \sum_{\omega \in A_i} X_i(\omega)$$
		
		
		\item Use the above expression and linearity of expectation to express $E(Y)$ in terms of $E(X_i(\omega))$ given that $\omega \in A_i$. 
		
	
		
		
		\item Let $A_i'$ be the event that $d_1$ is the lowest die out of $\{d_1,d_2,d_3,d_4\}$ and that $d_1 = i$. Let  $X_i'(\omega)=d_2+d_3+d_4$ if $\omega \in A_i'$ and let $X_i'(\omega)=0$ if $\omega \notin A_i'$. Find  $E(X_i'(\omega))$ given that $\omega \in A_i'$. That is, assuming that $i$ is the lowest die roll and $d_1 = i$, find the expected value of $X_i'$. Recall that $\forall \omega \in A_i', X_i'(\omega) = d_2(\omega)+d_3(\omega)+d_4(\omega)$, where each of $d_2,d_3,$ and $d_4$ is at least $i$.
	
		
	
		
		\item\label{given} Observe that the term $E(X_i(\omega))$ in $\sum_{\omega \in A_i} E(X_i(\omega))$ is a constant, since it is the weighted average of the values $\forall \omega \in A_i, X_i(\omega)$. Similarly the term $E(X_i'(\omega))$ in $\sum_{\omega \in A_i'} E(X_i'(\omega))$ is also a constant. We will not prove it at this time, but we will use the fact that for the expressions given above, $E(X_i'(\omega))<E(X_i(\omega))$ to show a lower bound on $E(Y)$. Given that $E(X_i'(\omega))<E(X_i(\omega))$, briefly explain why $\sum_{\omega \in A_i} E(X_i'(\omega))<\sum_{\omega \in A_i} E(X_i(\omega))$. 
		
		
		\item Recall that $A_i$ is the event that $\min\{d_1,d_2,d_3,d_4\}=i$. Show that $|A_i|= (6-i+1)^4-(6-i)^4$. 
		
	
		
		\item Explain why $\sum_{\omega \in A_i} E(X_i'(\omega)) = \left((6-i+1)^4-(6-i)^4\right)\cdot E(X_i'(\omega))$.
		
		
		
		\item Now show that
		
		\begin{align*}
			E(Y) & > \sum_{i=1}^6  \left(3\cdot\left(3+\frac{i}{2}\right)\right)\cdot\frac{(6-i+1)^4-(6-i)^4}{6^4}
		\end{align*} 
		which, if you plug into Wolfram alpha, is $>11.63$. 
		
	
		
		\item[Bonus:] In part \ref{given} we provide that $E(X_i)>E(X_i')$. That is, the average value of the highest $3$ dice of all the rolls in $A_i$ is higher than the average value of the highest $3$ dice of all the rolls in $A_i'$. Explain the idea behind why this is the case. You do not need to prove it, so you may use examples to help articulate it. Hint: $A_i'\subseteq A_i$. 
		

	\end{enumerate}

\end{question}
\newpage
\begin{question}
	\textbf{Bonus:} Michiel’s Craft Beer Company (MCBC) sells $n$ different brands of India
	Pale Ale (IPA). When you place an order, MCBC sends you one bottle of
	IPA, chosen uniformly at random from the $n$ different brands, independently
	of previous orders.\\
	\indent Simon places $m$ orders with MCBC. Define the random variable $X$ to
	be the total number of distinct brands that Simon receives. Determine the
	expected value $E(X)$ of $X$.\\
\indent	Hint: Use indicator random variables. Note that your answer will have two variables, $n$ and $m$, and thus might not simplify very much. 
\end{question}


\end{document}
